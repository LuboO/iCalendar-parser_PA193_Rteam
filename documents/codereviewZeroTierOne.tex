%%%%%%%%%%%%%%%%%%%%%%%%%%%%%%%%%%%%%%%%%
% Beamer Presentation
% LaTeX Template
% Version 1.0 (10/11/12)
%
% This template has been downloaded from:
% http://www.LaTeXTemplates.com
%
% License:
% CC BY-NC-SA 3.0 (http://creativecommons.org/licenses/by-nc-sa/3.0/)
%
%%%%%%%%%%%%%%%%%%%%%%%%%%%%%%%%%%%%%%%%%

%----------------------------------------------------------------------------------------
%	PACKAGES AND THEMES
%----------------------------------------------------------------------------------------

\documentclass{beamer}
\usepackage{listings}

\usepackage{color}

\definecolor{mygray}{rgb}{0.4,0.4,0.4}
\definecolor{myorange}{rgb}{1.0,0.4,0}

\lstset{
basicstyle=\tiny\sffamily\color{black},
commentstyle=\color{mygray},
frame=single,
keywordstyle=\color{blue},
showspaces=false,
showstringspaces=false,
stringstyle=\color{myorange},
tabsize=2,
language=C++
}


\mode<presentation> {

% The Beamer class comes with a number of default slide themes
% which change the colors and layouts of slides. Below this is a list
% of all the themes, uncomment each in turn to see what they look like.

%\usetheme{default}
%\usetheme{AnnArbor}
%\usetheme{Antibes}
%\usetheme{Bergen}
%\usetheme{Berkeley}
%\usetheme{Berlin}
%\usetheme{Boadilla}
%\usetheme{CambridgeUS}
%\usetheme{Copenhagen}
%\usetheme{Darmstadt}
%\usetheme{Dresden}
%\usetheme{Frankfurt}
%\usetheme{Goettingen}
%\usetheme{Hannover}
%\usetheme{Ilmenau}
%\usetheme{JuanLesPins}
%\usetheme{Luebeck}
\usetheme{Madrid}
%\usetheme{Malmoe}
%\usetheme{Marburg}
%\usetheme{Montpellier}
%\usetheme{PaloAlto}
%\usetheme{Pittsburgh}
%\usetheme{Rochester}
%\usetheme{Singapore}
%\usetheme{Szeged}
%\usetheme{Warsaw}

% As well as themes, the Beamer class has a number of color themes
% for any slide theme. Uncomment each of these in turn to see how it
% changes the colors of your current slide theme.

%\usecolortheme{albatross}
%\usecolortheme{beaver}
%\usecolortheme{beetle}
%\usecolortheme{crane}
%\usecolortheme{dolphin}
%\usecolortheme{dove}
%\usecolortheme{fly}
%\usecolortheme{lily}
%\usecolortheme{orchid}
%\usecolortheme{rose}
%\usecolortheme{seagull}
%\usecolortheme{seahorse}
%\usecolortheme{whale}
%\usecolortheme{wolverine}

%\setbeamertemplate{footline} % To remove the footer line in all slides uncomment this line
%\setbeamertemplate{footline}[page number] % To replace the footer line in all slides with a simple slide count uncomment this line

\setbeamertemplate{navigation symbols}{} % To remove the navigation symbols from the bottom of all slides uncomment this line
}


\usepackage[utf8]{inputenc}
\usepackage{graphicx} % Allows including images
\usepackage{booktabs} % Allows the use of \toprule, \midrule and \bottomrule in tables

%----------------------------------------------------------------------------------------
%	TITLE PAGE
%----------------------------------------------------------------------------------------

\title[ZeroTierOne code review]{ZeroTierOne code review} % The short title appears at the bottom of every slide, the full title is only on the title page

\author[R Team]{Ondrej Mosnáček , Lenka Kuníková , Ľubomír Obrátil} % Your name
\date{17.12.2015} % Date, can be changed to a custom date

\begin{document}

\begin{frame}
\titlepage % Print the title page as the first slide
\end{frame}

%----------------------------------------------------------------------------------------
%	PRESENTATION SLIDES
%----------------------------------------------------------------------------------------

\begin{frame}
\frametitle{ZeroTierOne}
\begin{itemize}
\item Network virtualisation service
\item Provides encryption, authentication, access control, network management
\item Source \url{https://github.com/zerotier/ZeroTierOne}
\end{itemize}
\end{frame}

\begin{frame}
\frametitle{Cppcheck}
\begin{itemize}
\item 4 errors
\item 80 warnings
\item 218 style warnings
\item 3 performance warnings
\item 1 portability warning
\end{itemize}
\end{frame}

\begin{frame}[fragile]
\frametitle{Cppcheck - errors}
\begin{itemize}
\item Exception thrown in function declared not to throw exceptions
\begin{itemize}
\item 3x - use of their own function snprintf() which throws an exception if the output buffer is too short
\end{itemize}
\item Used file that is not opened.
\end{itemize}
\begin{lstlisting}
fclose(pkg);
FILE *bash = fopen(bashp,"w");
if (!bash) {
fclose(pkg);
...
}
\end{lstlisting}
\end{frame}

\begin{frame}[fragile]
\frametitle{Cppcheck - warnings}
\begin{itemize}
\item \%d in format string requires 'int' but the argument type is 'unsigned int'
\item Member variable not initialized in the constructor
\item 'operator=' should check for assignment to self 
\item Repositioning operation performed on a file opened in append mode has no effect
\begin{lstlisting}
FILE *f = fopen(reportSavePath.c_str(),"a");
		if (!f)
			return;
		fseek(f,0,SEEK_END);
\end{lstlisting}
\item (Portability) Using memset() on union which contains a floating point number
\item (Performance)  Variables are assigned in constructor body
\end{itemize}
\end{frame}

\begin{frame}[fragile]
\frametitle{Cppcheck - style}
\begin{itemize}
\item The scope of variables can be reduced
\item Variable is assigned a value that is never used.
\item Unused variables
\item missing copy constructor in a class that contains a pointer to allocated memory
\item The variable 'k' is used as an array index before it is checked that is within limits
\begin{lstlisting}
while ((!ip[k])&&(k < 15)) ++k;
\end{lstlisting}
\item Consecutive throw and return statements are unnecessary
\begin{lstlisting}
	if (ioctl(sock,SIOCSIFHWADDR,(void *)&ifr) < 0) {
		...
		throw std::runtime_error("unable to configure TAP hardware (MAC) address");
		return;
	}
\end{lstlisting}
\end{itemize}
\end{frame}

\begin{frame}
\frametitle{MSVC}
\begin{itemize}
\item 1 error - use of deprecated function inet\_addr()
\item Warnings:
\begin{itemize}
\item conversion from unsigned int to char
\item nonstandard extension used: non-constant aggregate initializer
\item const object should be initialized
\item unreferenced function parameter
\item assignment within conditional expression
\end{itemize}
\end{itemize}
\end{frame}

%------------------------------------------------


\end{document} 