%%%%%%%%%%%%%%%%%%%%%%%%%%%%%%%%%%%%%%%%%
% Beamer Presentation
% LaTeX Template
% Version 1.0 (10/11/12)
%
% This template has been downloaded from:
% http://www.LaTeXTemplates.com
%
% License:
% CC BY-NC-SA 3.0 (http://creativecommons.org/licenses/by-nc-sa/3.0/)
%
%%%%%%%%%%%%%%%%%%%%%%%%%%%%%%%%%%%%%%%%%

%----------------------------------------------------------------------------------------
%	PACKAGES AND THEMES
%----------------------------------------------------------------------------------------

\documentclass{beamer}
\usepackage{listings}

\usepackage{color}

\definecolor{mygray}{rgb}{0.4,0.4,0.4}
\definecolor{myorange}{rgb}{1.0,0.4,0}

\lstset{
basicstyle=\tiny\sffamily\color{black},
commentstyle=\color{mygray},
frame=single,
keywordstyle=\color{blue},
showspaces=false,
showstringspaces=false,
stringstyle=\color{myorange},
tabsize=2,
language=C++
}


\mode<presentation> {

% The Beamer class comes with a number of default slide themes
% which change the colors and layouts of slides. Below this is a list
% of all the themes, uncomment each in turn to see what they look like.

%\usetheme{default}
%\usetheme{AnnArbor}
%\usetheme{Antibes}
%\usetheme{Bergen}
%\usetheme{Berkeley}
%\usetheme{Berlin}
%\usetheme{Boadilla}
%\usetheme{CambridgeUS}
%\usetheme{Copenhagen}
%\usetheme{Darmstadt}
%\usetheme{Dresden}
%\usetheme{Frankfurt}
%\usetheme{Goettingen}
%\usetheme{Hannover}
%\usetheme{Ilmenau}
%\usetheme{JuanLesPins}
%\usetheme{Luebeck}
\usetheme{Madrid}
%\usetheme{Malmoe}
%\usetheme{Marburg}
%\usetheme{Montpellier}
%\usetheme{PaloAlto}
%\usetheme{Pittsburgh}
%\usetheme{Rochester}
%\usetheme{Singapore}
%\usetheme{Szeged}
%\usetheme{Warsaw}

% As well as themes, the Beamer class has a number of color themes
% for any slide theme. Uncomment each of these in turn to see how it
% changes the colors of your current slide theme.

%\usecolortheme{albatross}
%\usecolortheme{beaver}
%\usecolortheme{beetle}
%\usecolortheme{crane}
%\usecolortheme{dolphin}
%\usecolortheme{dove}
%\usecolortheme{fly}
%\usecolortheme{lily}
%\usecolortheme{orchid}
%\usecolortheme{rose}
%\usecolortheme{seagull}
%\usecolortheme{seahorse}
%\usecolortheme{whale}
%\usecolortheme{wolverine}

%\setbeamertemplate{footline} % To remove the footer line in all slides uncomment this line
%\setbeamertemplate{footline}[page number] % To replace the footer line in all slides with a simple slide count uncomment this line

\setbeamertemplate{navigation symbols}{} % To remove the navigation symbols from the bottom of all slides uncomment this line
}


\usepackage[utf8]{inputenc}
\usepackage{graphicx} % Allows including images
\usepackage{booktabs} % Allows the use of \toprule, \midrule and \bottomrule in tables

%----------------------------------------------------------------------------------------
%	TITLE PAGE
%----------------------------------------------------------------------------------------

\title[Team P code review]{JSON Floorball parser code review} % The short title appears at the bottom of every slide, the full title is only on the title page

\author[R Team]{Ondrej Mosnáček , Lenka Kuníková , Ľubomír Obrátil} % Your name
\date{17.12.2015} % Date, can be changed to a custom date

\begin{document}

\begin{frame}
\titlepage % Print the title page as the first slide
\end{frame}

%----------------------------------------------------------------------------------------
%	PRESENTATION SLIDES
%----------------------------------------------------------------------------------------

\begin{frame}
\frametitle{Style}

\begin{itemize}
\item \textbf{General} -- inconsistent method naming
\item \textbf{General} -- hardcoded strings, lots of magic constants depending on these strings
\item \textbf{General} -- using exit(1) inside functions -- code can't be modified to parse multiple files
\item \textbf{Team.cpp, Player.cpp} -- code duplicity
\item \textbf{Player.h} -- public methods that should be private
\item \textbf{Team.cpp} -- not using initializer section
\item \textbf{Team.cpp} -- double comparison in if statements
\item \textbf{main.cpp} -- hardcoded output file name
\end{itemize}

\end{frame}

%------------------------------------------------

\begin{frame}[fragile]
\frametitle{Style issues examples}

\begin{lstlisting}
/* Inconsistent names */
long long skipWhiteSpace();             // Match.h(33)
static bool IsForbiddenUTF8Char(byte b) // UTF.h(27)

/* Hardcoded strings, magic numbers */ // Player.cpp(142)
int startGFDef = skipToChar(comma1 + 1 , '"' , player) + 1; 
if(player.compare(startGFDef , 10 , "GFbyPlayer") != 0 || player[startGFDef + 10] != '"')
{cerr << "Json GFbyPlayer name not correct" << endl; exit(1);}
int semicolon2 = skipToChar(startGFDef + 11 , ':' , player);
int endGF = checkInt(semicolon2 + 1 , player , m_gf);
int comma2 = skipToChar(endGF , ',' , player);

/* Code duplicity */
int checkString(int index , const string &buffer , string &variable) // Team.cpp(101)
int checkString(int index , const string &player , string &variable) // Player.cpp(92)

/* Double comparison */ // Team.cpp(143)
if(player[index + 1] == '/' || player[index + 1] == '\b' || player[index + 1] == '\f' ||  
   player[index + 1] == '\f' || player[index + 1] == '\n' || player[index + 1] == '\r' ||
   player[index + 1] == '\t' || player[index + 1] == '"' || player[index + 1] == '\\')
\end{lstlisting}

\end{frame}

%------------------------------------------------

\begin{frame}
\frametitle{Performance/Portability issues}

\begin{itemize}
\item \textbf{General} -- \#pragma once is not standard
\item \textbf{General} -- whole input is loaded into string first
\item \textbf{General} -- too many substrings are unnecessary copied
\item \textbf{Team.h} -- wrong case in \#include, wouldn't build on *NIX systems
\item \textbf{Team.h} -- wrong use of quotes vs angle brackets in \#include
\item \textbf{Team.cpp} -- use of std::stoi -- part of C++11, yet project report doesn't mention it
\end{itemize}

\end{frame}

%------------------------------------------------

\begin{frame}[allowframebreaks=0.8]
\frametitle{Bugs and crashes}

\begin{itemize}
\item \textbf{UTF.h} -- some UTF-8 byte sequences that should be rejected are not (e. g. codepoint U+0024 must be encoded as 1 byte, but the parser also accepts the two- or more-byte version)
\item \textbf{UTF.h} -- IsForbiddenUTF8Char -- unclear purpose; \textbf{all comparisons here are always false} (char variable compared to a constant $>$ 0x7F)
\item \textbf{Team.cpp} -- parse, findAndCheck -- end of string is not checked anywhere, causes unhandled exception/crash on some inputs
\item \textbf{Player.cpp, Team.cpp} -- checkString -- no bound checking for string, certain inputs will cause reading after the buffer and thus crashing
\item \textbf{Player.cpp, Team.cpp} -- checkString -- wrong character escaping -- will allow e. g. \textbackslash 0 or \textbackslash n in strings
\item \textbf{Player.cpp, Team.cpp} -- checkString -- unescaped / is rejected, yet JSON allows it
\item \textbf{Player.cpp, Team.cpp} -- checkString -- \textbackslash uXXXX escape allowed by JSON is not validated (e. g. allows "\textbackslash uHello"; should be only hexadecimal digits)
\item \textbf{Player.cpp} -- Parse -- will reject any valid input if it doesn't have whitespace before curly bracket
\item \textbf{main.cpp} -- bad\_alloc exception is not handled while parsing
\end{itemize}

\end{frame}

%------------------------------------------------

\begin{frame}[fragile]
\frametitle{Pieces of malformed input}

\begin{lstlisting}[stringstyle=\color{black}]
/* Invalid UTF8 (escaped characters) */
"Name": "\xf0\x82\x82\xac Doe",

/* Zero in name (escaped zero) */
"Name": "\0 Doe",

/* Out of bounds reading */
"Name": "\\

/* Rejected valid input */
"GPbyPlayer": 2},

/* Newline in name */
"Name": "\
",
\end{lstlisting}

\end{frame}

\begin{frame}{Tools \& methods}
\begin{itemize}
\item \textbf{cppcheck} -- detected minor issues (e. g. the duplicate comparison)
\item \textbf{PREfast} -- detected nothing
\item \textbf{zzuf, radamsa} (fuzzers) -- mostly generated invalid files, did not cause crashes
\item \textbf{custom fuzzing}
	\begin{itemize}
	\item cut off bytes from the end of a valid file one-by-one
	\item this approach helped discover some crashes/memory corruptions
	\end{itemize}
\item \textbf{valgrind} -- used together with fuzzers to detect memory issues
\item \textbf{gcc, clang, MSVC} -- all detected the tautological comparisons with all warnings enabled + some minor issues
\item \textbf{manual review} -- most of the issues were found by manual review
\end{itemize}
\end{frame}

%------------------------------------------------

\begin{frame}
\frametitle{The End}

\begin{center}
\begin{huge}
Thank you for your attention. \\
Questions?
\end{huge}
\end{center}

\end{frame}

\end{document} 